\documentclass{article}
\usepackage{amsmath}
\usepackage{amssymb}
\usepackage[margin=1in]{geometry}
\usepackage{array} 

\begin{document}

\textbf{Exercice 6}

\vspace{0.5cm}

Nous affirmons que l’hypothèse manquante est que $\mathcal{R}$ soit total.

\vspace{0.5cm}

\textbf{Démonstration} Soit \(S\) un ensemble et \(\mathcal{R} \subseteq S^2\) une relation symétrique, transitive, et totale. Nous 

voulons démontrer que \( \mathcal{R} \) est réflexive. Il faut montrer que $ ( \forall a \in S \mid a \mathcal{R} a)$.

Soit \(a \in S\). Comme \( \mathcal{R} \) est totale, par définition, il existe un \(b \in S\) tel que 

$$ \langle a, b \rangle \in \mathcal{R} $$

Si $a = b$ on a fini $ \langle a,a \rangle \in \mathcal{R}$ implique que $ \mathcal{R}$ est reflexive.

Si $a \neq b$ alors, comme $ \mathcal{R} $ est symétrique on a $(\forall a,b \in S \mid \langle a,b \rangle \in \mathcal{R} \implies \langle b,a \rangle \in \mathcal{R} )$ qui implique

$$\langle b,a \rangle \in \mathcal{R}.$$

Comme $\mathcal{R}$ est aussi transitive $ ( \forall a, b, c, \in S \mid \langle a,b \rangle \in \mathcal{R} \land \langle b,c \rangle \in \mathcal{R} \implies \langle a,c \rangle \in \mathcal{R} )$. Donc, on peut 

deduire $\langle a,b \rangle \in \mathcal{R} \land \langle b,a \rangle \in \mathcal{R}$ seulement si

$$\langle a,a \rangle \in \mathcal{R}.$$

Puisque pour tout $ a $ nous pouvons toujours épuiser les cas en considérant $a = b$ et $a \neq b$ dans nos deux 

cas, et dans tous les cas, nous avons montré que $( \forall a \in S \mid \langle a,a \rangle \in \mathcal{R} ) $. Ainsi $ \mathcal{R}$ est reflexive.

$\hfill \square$

\end{document}
