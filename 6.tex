\documentclass{article}
\usepackage{amsmath}
\usepackage{amssymb}
\usepackage[margin=1in]{geometry}
\usepackage{array} 

\begin{document}

\textbf{Exercice 6}

\vspace{0.5cm}

Nous affirmons que l’hypothèse manquante est que $\mathcal{R}$ soit total.

\vspace{0.5cm}

\textbf{Démonstration} Soit $S$ un ensemble $\mathcal{R} \subseteq S^2 $ une relation symétrique, transitive et total. Nous voulons 

demontrer que \( \mathcal{R} \) est reflexive.

Soit $a \in S $ et comme $ \mathcal{R} $ est total il y a une $b \in S$ tel que 
$$ \langle a, b \rangle \in \mathcal{R} $$

Si $a = b$ on a fini $ \langle a,a \rangle \in \mathcal{R}$ implique que $ \mathcal{R}$ est reflexive.

Si $a \neq b$ alors, comme $ \mathcal{R} $ est symétrique on a

$$\langle b,a \rangle \in \mathcal{R}.$$

Comme $\mathcal{R}$ est aussi transitive on a si $ \langle a,b \rangle \in \mathcal{R} \land \langle b,a \rangle \in \mathcal{R}$

$$\langle a,a \rangle \in \mathcal{R}$$

Donc, dans tous les cas $( \forall a \in S \mid \langle a,a \rangle \in \mathcal{R} ) $. Ainsi $ \mathcal{R}$ est reflexive.

$\hfill \square$

\end{document}
