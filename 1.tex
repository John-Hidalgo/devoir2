\documentclass{article}
\usepackage{amsmath}
\usepackage{amssymb}
\usepackage[margin=1in]{geometry}
\usepackage{array} 

\newcommand{\role}{\mbox{RÔLE}}
\newcommand{\attribut}{\mbox{ATTRIBUT}}
\newcommand{\boite}{\mbox{BOÎTE}}
\newcommand{\compose}{\circ}
\newcommand{\tuple}[1]{\ensuremath{\left\langle #1 \right\rangle}}

\begin{document}

\textbf{Exercice 1}

a) $\text{Dom}(\textit{espèce} \cap \role \times \{ \textsf{munceau} \} )$

\vspace{0.5cm}

b) $ \text{Im}( \text{Im}( \textit{espèce} \cap \{ \textsf{gnome} \} \times \attribut ) \times \mathbb{N} \setminus \{ 0,1,2,3,4,5 \} ) $ 

ou

$\{ n \in N \mid (\exists a \in \attribut\ \mid \langle \textsf{gnome},a  \rangle \in \textit{espèce} \land \langle a,n \rangle \in \textit{plus} \land n > \text{len}(\textsf{gnome}) )\}$

\vspace{0.5cm}

c) $\textit{espèce} \compose \textit{plus}$ je vais le dire en anglais. I think for this one it's actually $\textit{espèce} \cup \textit{plus}$ I thought 
because the prof mentions rule 2 that we should only look at pairs that have corresponding plus cartes which we can handle composing them. 
But the rule doesn't actually say that the cards need to have a plus card only that they could be enriched by one if present in a players hand. 
Then again again maybe she means in a pack of cards all the espece cards need to have at least one plus card. What do you think?

\vspace{0.5cm}

d) Dom(\textit{plus})

\vspace{0.5cm}

e) $ \text{Im}(\textit{espèce} \cap \{ \textsf{gnome}, \textsf{lionceau} \} \times \attribut)$

\vspace{0.5cm}

f) $ \text{Im}(\textit{espèce})^{c} $

\vspace{0.5cm}

g) Si nous imaginons que les cartes d'espèces sont achetées en paquets, $\textit{espèce}^{c}$ sont les cartes espèces 

qu'il n'est pas possible d'acheter.

\vspace{0.5cm}

h) $( \forall \tuple{r,a} \in \textit{espèce} \mid ( \exists \tuple{a,n} \in \textit{plus} \mid n > \text{len}(r) ))$

\vspace{0.5cm}

i) $ (  \forall \tuple{a,n} \in \textit{plus} \mid |\{ r \in \role \mid \tuple{r,a} \in \textit{espèce} \land \text{len}(r) < m \}| \geq 0.1 | \role | ) $

\vspace{0.5cm}

j) $( \exists R \subset \role \mid |R| > 2 \land ( \forall r \in R \mid \text{len}(r) \leq 15 \land ( \exists r^{\prime} \in \role \mid \text{len}(r - r^{\prime}) > 15 ) ) ) $

\vspace{0.5cm}

k) Les éléments des ensembles sont uniques, donc les rôles et les attributs sont uniques, ce qui conduit à

des paires uniques qui sont les cartes. Techniquement, vous pourriez affirmer que vous pouvez toujours 

former des paires en double, mais les produits cartésiens d'ensembles sont également des ensembles, donc 

encore une fois, ils sont uniques.



\end{document}
