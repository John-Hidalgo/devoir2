\documentclass{article}
\usepackage{amsmath}
\usepackage{amssymb}
\usepackage[margin=1in]{geometry}
\usepackage{array}

\begin{document}

\textbf{Exercice 4}

\vspace{0.5cm}

A)

1) Oui, $g$ est déterministe.

\textbf{Démonstration} Soient $x, y, y^{\prime} \in \mathbb{R}$, il faut montrer que $g(x) = y \land g(x) = y^{\prime}$ implique que $y = y^{\prime}$.

Supposons que  
$$ g(x) = y \land g(x) = y^{\prime}. $$

Par définition de $g$ et par la propriété transitive de l'égalité 
$$ y = 42 + 5x \land y^{\prime} = 42 + 5x \implies y = y^{\prime}. $$

$\hfill \square$

2) Oui, $g$ est totale et ainsi une fonction.

\textbf{Démonstration} Soit $x \in \mathbb{R}$, il faut trouver un $y \in \mathbb{R}$ tel que $g(x) = y$.

Les nombres réels sont fermés par multiplication et addition. En particulier, cela signifie que

$$ 42 + 5x  \in \mathbb{R}. $$

Il suffit de prendre $y = 42 + 5x$ pour que $y = g(x)$.

$\hfill \square$

3) Oui, $g$ est une fonction injective.

\textbf{Démonstration} Soient $x, x^{\prime}, y \in \mathbb{R}$, il faut montrer que $g(x) = y \land g(x^{\prime}) = y$ implique que $x = x^{\prime}$.

Supposons que  
$$ g(x) = y \land g(x^{\prime}) = y. $$

Évaluons $g$, et puisque l'égalité est transitive 
$$ y = 42 + 5x \land y = 42 + 5x^{\prime} \implies 42 + 5x = 42 + 5x^{\prime} \implies x = x^{\prime}. $$

$\hfill \square$

4) Oui, $g$ est une fonction surjective.

\textbf{Démonstration} Soit $y \in \mathbb{R}$, il faut trouver un $x \in \mathbb{R}$ tel que $g(x) = y$. Les nombres réels sont fermés 

par multiplication et soustraction. Donc, pour notre $y$

$$ \frac{y - 42}{5} \in \mathbb{R}. $$

Prenons maintenant $x = \frac{y - 42}{5}$ et calculons

$$ g(x) = g\left(\frac{y - 42}{5}\right) = 5\left(\frac{y - 42}{5}\right) + 42 = y - 42 + 42 = y. $$

Donc, $g$ est surjectif.

$\hfill \square$

5) Oui, $g$ est bijective car elle est injective et surjective.

\vspace{0.5cm}

B) $g^{-1} : \mathbb{R} \longrightarrow \mathbb{R}$ définie par $x \longmapsto \frac{x - 42}{5}$.

\end{document}
