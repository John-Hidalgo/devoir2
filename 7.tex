\documentclass{article}
\usepackage{amsmath}
\usepackage{amssymb}
\usepackage[margin=0.5in]{geometry}
\usepackage{array} 

\begin{document}

\textbf{Exercice 7}

\vspace{0.5cm}

\textbf{Démonstration} Soit $S$ un ensemble $ \mathcal{R} \subseteq \mathcal{L} \subseteq S^2 $ et \( \mathcal{L} \) une relation asymétrique. Nous voulons 

demontrer que \( \mathcal{R} \) est asymétrique. Nous devons montrer que $\left(\forall  a, b \in S \mid \,  \langle a, b \rangle \in \mathcal{R} \implies \langle b,a \rangle \notin \mathcal{R} \right)$.

À cette fin, soient $a, b \in S$ et supposons
$$ \langle a, b \rangle \in \mathcal{R} $$

alors, comme $ \mathcal{R} \subseteq \mathcal{L} $ par définition ça veut dire $\left(\forall a, b \mid \,  \langle a, b \rangle \in \mathcal{R} \implies \langle a, b \rangle \in \mathcal{L} \right)$, nous donnant 

$$ \langle a, b \rangle \in \mathcal{L} $$

Car $\mathcal{L}$ est asymétrique, cela signifie $\left(\forall a, b \in S \mid \,  \langle a, b \rangle \in \mathcal{L} \implies \langle b,a  \rangle \notin \mathcal{L} \right)$. Donc,

$$\langle b, a \rangle \notin \mathcal{L}$$

Par la contrapositive de l'implication pour la définition du sous-ensemble nous avons que 

$\left(\forall  a, b \mid \,  \langle a, b \rangle \notin \mathcal{L} \implies \langle a, b \rangle \notin \mathcal{R} \right)$ et ainsi

$$\langle b,a \rangle \notin \mathcal{R}.$$

Puisque $a$ et $b$ ont été choisis arbitrairement, nous avons montré que $\left(\forall  a, b \in S \mid \,  \langle a, b \rangle \in \mathcal{R} \implies \langle b,a \rangle \notin \mathcal{R} \right)$ 

et donc \( \mathcal{R} \) est asymétrique.


$\hfill \square$

\end{document}
