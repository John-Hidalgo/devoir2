\documentclass{article}
\usepackage{amsmath}
\usepackage{amssymb}
\usepackage[margin=1in]{geometry}
\usepackage{array}
\newcommand{\tuple}[1]{\ensuremath{\left\langle #1 \right\rangle}}

\begin{document}

\textbf{Exercice 5}

\vspace{0.5cm}

A)

1) Oui, $h$ est déterministe.

\textbf{Démonstration} Soient $a, b, b^{\prime} \in \mathbb{R}$, il faut montrer que $b = b^{\prime}$ si $a \hspace{.1cm} h \hspace{.1cm} b \land a \hspace{.1cm} h \hspace{.1cm} b^{\prime}$.

Supposons que  
$$ a \hspace{.1cm} h \hspace{.1cm} b \land a \hspace{.1cm} h \hspace{.1cm} b^{\prime}. $$

Par définition de $h$ et par la propriété transitive de l'égalité :
$$ b = 2a^2 + 42 \land b^{\prime} = 2a^2 + 42 \implies b = b^{\prime}. $$

Car $a, b, b^{\prime}$ ont été choisis arbitrairement nous l'avons montré $ ( \forall a, b, b^{\prime} \in \mathbb{R} \mid a \hspace{.1cm} h \hspace{.1cm} b \land a \hspace{.1cm} h \hspace{.1cm} b^{\prime} \implies b = b^{\prime}). $

Donc $g$ est déterministe. 

$\hfill \square$

2) Oui, $h$ est total et ainsi une fonction.

\textbf{Démonstration} Soit $a \in \mathbb{R}$, il faut trouver un $b \in \mathbb{R}$ tel que $a \hspace{.1cm} h \hspace{.1cm} b$.

Les nombres réels sont fermés par multiplication et addition. En particulier, cela signifie que pour $a$, 
$$ a^2 + 42  \in \mathbb{R}. $$

Prenons simplement $b = a^2 + 42$, et alors $a \hspace{.1cm} h \hspace{.1cm} b$. Puisque $a$ a été choisi arbitrairement on a que

$( \forall a \in \mathbb{R} \mid ( \exists b \in \mathbb{R} \mid a \hspace{.1cm} h \hspace{.1cm} b ) ).$ Donc $h$ est total et ainsi une fonction.

$\hfill \square$

3) Non, $h$ n'est pas une fonction injective.

\textbf{Démonstration} Voici un contre-exemple : $2$, $-2$ et $44$ sont des nombres réels pour lesquels $2^2 + 42 = 44$ 

et $(-2)^2 + 42 = 44$, mais pourtant $2 \neq -2$. Donc, comme on a que $ ( \exists a, a^{\prime}, b \in \mathbb{R} \mid ( a \hspace{.1cm} h \hspace{.1cm} b \land a^{\prime} \hspace{.1cm} h \hspace{.1cm} b) \land a \neq a^{\prime} )) $ 

qui est la négation de la définition de l'injectivité, $h$ n'est pas injetive.

$\hfill \square$

4) Non, $h$ n'est pas non plus surjective.

\textbf{Démonstration} Il faut trouver un $b \in \mathbb{R}$ tel que $\forall a \in \mathbb{R}$, $b \neq a^2 + 42$. Prenons $b=0$. C'est un 

nombre, or pour tout réel $a$, on a $a^2 > 0 \implies 0 < a^2 + 42$. Donc $( \exists b \in \mathbb{R} \mid ( \forall a \in \mathbb{R} \mid \tuple{a,b} \notin h ))$ ce qui 

est la négation de la définition de la surjectivité, $h$ n'est pas surjective.

$\hfill \square$

5) Non, $h$ n'est pas bijective car il n'est ni injective ni surjective.

\vspace{0.5cm}

\end{document}
